\section{Well-posedness of the optimal control problem for $\gamma=0$ without norm-constraint on the control}
\label{wp2}
%!TEX root = kdv.tex
In this section we prove existence of optimal controls in the non-viscous case $\gamma=0$ without norm-constraints on the optimal control. The proof is based on the knowledge that the boundedness of the state $y$ in the $L^4(I\times\Omega))$-norm implies an a priori estimate of $y$ in $\mathcal B$ and implies $y\partial_xy\in\Hm1$. Therefore, we add the $L^4(I\times\Omega)$-norm as an additional regularization term to objective functional of \eqref{cost} and use the non state reduced approach from \cite{lions1985control}. In \cite{lions1985control}, the state equation is considered as an equality constraint in the pair $(y,u)$ in contrast to the reduced approach from the previous sections.\\
In the following we consider an optimal control problem of the form
\begin{multline}
\min_{u \in \M\,,y\in \tilde Y}J(y,u)=\frac{1}{2}\left(\norm{\chi_{\Omega_{o}}y - z_1}_{L^2(I\times \Omega_{o})}^2+\|\chi_{\Omega_{o}}y(T)-z_2\|_{L^2(\Omega_{o})}^2\right) \\+ \alpha \norm{u}_{\mathcal{M}(\Omega, L^{2}(I))} + \frac{\eta}{4} \norm{y}_{L^4(I\times \Omega)}^4.
\label{cost2}
\end{multline}
subject to \eqref{kdvcontrol1}-\eqref{kdvcontrol3} with $\gamma=0$. The state space is chosen as $\tilde Y = L^4(I,L^4(\Omega))\cap H^1(I,\mathcal V^\ast)$.  The $L^4(I,L^4(\Omega))$-norm is chosen since it guarantees the following regularity of the non-linearity.
\begin{lemma}\label{lemyyxL4}
Let $y \in L^4(I,L^4(\Omega))$. Then $y \partial_x y \in L^2(I,H^{-1}(\Omega))$.\\
\end{lemma}
\begin{proof}
{\color{red}The proof is based on similar arguments as in the proof of Lemma \ref{lemyyx2}.}
\qquad\end{proof}

Therefore the following definition is reasonable.
\begin{definition}\label{statecontrolpair}
Let $(y,u)$ $\in \tilde Y\times \M$ such that $y(T,\cdot)\in L^2(\Omega)$. They are called a \textit{state-control pair} if they satisfy the equation
\begin{equation}\label{weakformkdv L4}
\int_0^T(y,\phi)_{L^2(\Omega)}~\mathrm dt+(y(T),p_T)_{L^2(\Omega)}=\int_0^T\langle u-y\partial_xy,p\rangle_{H^{-1}(\Omega),H^1_0(\Omega)}~\mathrm dt+(y_0,p(0))_{L^2(\Omega)}
\end{equation}
for all $(\phi,p_T) \in L^{4/3}(I,L^2(\Omega))\times L^2(\Omega)$, where $p = p(\phi,p_T)\in \mathcal B$ is the mild solution of \eqref{kdvlinnonhomdual1}-\eqref{kdvlinnonhomdual3}.
\end{definition}

Next we show that the existence of a control-state pair implies the time-global existence of $y$ in $\mathcal B$.
\begin{proposition}\label{statecontrolestimate}
 Let $(y,u)$ $\in \tilde Y\times \M$ be a state-control pair. Then $ y\in \mathcal{B} \cap H^{1}(I,\mathcal{V}^{*})$ is a weak solution of \eqref{kdvcontrol1} - \eqref{kdvcontrol3} in the sense of Definition~\ref{defnlkdv} for $\gamma = 0$ and satisfies the global estimate
 \be
 \|y\|_{\mathcal{B}}\leq C(T,L)\,\left(\|f\|_{\Hm1}+\|y_0\|_{L^2(\Omega)} + \|y\|_{L^4(I,L^4(\Omega))}^{2}\right)
 \ee
 and
 \[
 \|\partial_{t}y\|_{L^2(I,\mathcal{V}^{*})}\leq C(T,L,f,y_0).
 \]
\end{proposition}
\begin{proof}
For the first part of the estimate, one proceed in a similar manner as in Appendix \ref{sec:linear-estimates} but this time for the non-linear \KdV equation. The critical term is then the non-linearity, the rest is identical. Let $y\in \mathcal C(\bar I,\mathcal D(A))~\cap~\mathcal C^1(\bar I,L^2(\Omega))$ be the classical solution of \eqref{kdvcontrol1} - \eqref{kdvcontrol3} for smooth data ($\mathcal C^\infty$). We multiply \eqref{kdv1} which holds in $L^2(\Omega)$ for a.e. $t\in I$ with $y$ and get for the nonlinear term
\[
\int_0^Ly^2\partial_xy~\mathrm dx=-2\int_0^Ly^2\partial_xy~\mathrm dx.
\]
Due to the boundary conditions, this term vanishes. Next, we test with $xy$ and see
\[
\int_0^Lxy^2\partial_x y~\mathrm dx = -\int_0^L y^3+2xy^2\partial_x y~\mathrm dx.
\]
This yields
\be
\nonumber
\int_0^Lxy^2\partial_x y~\mathrm dx = -\frac{1}{3}\int_0^L y^3~\mathrm dx.
\ee
This additional term is then treated in the same manner as the source term $f$ in Appendix \ref{sec:linear-estimates}. This explains the necessity of the boundedness of $\|y\|_{L^3(I\times\Omega)}$, which is guaranteed by the fact that $y \in L^4(I,L^4(\Omega))$. The estimate for the time derivative $\|\partial_{t}y\|_{L^2(I,\mathcal{V}^{*})}$ follows from \eqref{weakformkdv L4}, Lemma~\ref{lemyyxL4} and the global estimate for $\norm{y}_{\mathcal{B}}$.
\qquad\end{proof}

Now we are in the position to show the existence of an optimal control.
\begin{proposition}
There exists a solution $(\bar y, \bar u)\in \tilde Y\times \M $ to the optimal control problem \eqref{cost2}.
\end{proposition}
\begin{proof}
Since the cost functional $J$ is positive it has an infimum $\bar J$. Moreover we already have shown in Proposition~\ref{localposedness} existence of a control-state pair $(y,u)\in \tilde Y\times \M$ (for small data, see Remark~\ref{rmkUad}). Thus there exists a minimizing sequence of state-control pairs $(y_n,u_n) \in \tilde Y\times \M$ such that $J(y_n, u_n) \rightarrow \bar J$ as $n \rightarrow \infty$. Furthermore there is a $\varepsilon>0$ such that
\be
\varepsilon+J(0,0)\geq \alpha \norm{u_n}_{\M} + \frac{\eta}{4} \norm{y}_{L^4(I\times \Omega)}^4
\ee
for $n$ large enough. Therefore the sequence $(y_n,u_n)$ is bounded in $L^4(I,L^4(\Omega))\times \M$. Thanks to Proposition~\ref{statecontrolestimate}, we also have $(y_n,u_n)$ is bounded in $\mathcal{B}\cap H^{1}(I, \mathcal{V}^{*})\times \M$. This implies the existence of an element $(\bar y, \bar u)\in \mathcal{B}\cap H^{1}(I, \mathcal{V}^{*})\times \M$ and a state-control subsequence $(y_{n_k},u_{n_k})$ such that $u_{n_k}$ converges in the weak-$*$ topology of $\M$ towards $\bar u$ and $y_{n_k}$ converges in the weak-$\ast$ topology of $\mathcal{B}\cap H^{1}(I, \mathcal{V}^{*})$ towards $\bar y$. Moreover the convergence $y_{n_k}(T) \rightharpoonup \bar y(T)$ in $L^2(\Omega)$ is ensured. With the Aubin-Lions lemma, we get also
\be
\nonumber
y_{n_k} \rightarrow \bar y\quad\text{in}\quad L^2(I,\mathcal C_0(\Omega)).
\ee
Next we check that $(\bar y, \bar u)$ is a state-control pair according to Definition~\ref{statecontrolpair}. The convergence of the linear terms in \eqref{weakformkdv L4} is obvious. The convergence of the nonlinear terms is due to strong convergence of $y_{n_k}$ towards $\bar y$. %, as
%\begin{multline*}
%\int_0^T\langle y_{n_k}\partial_xy_{n_k}-\bar y\partial_x\bar y,p\rangle_{H^{-1}(\Omega),H^1_0(\Omega)}~\mathrm dt=-\int_0^T(y_{n_k}^2-\bar y^2,\partial_x p)_{L^2(\Omega)}~\mathrm dt\\
%\leq\|y_{n_k}-\bar y\|_{L^2(I,\mathcal C_0(\Omega))}\|y_{n_k}+\bar y\|_{\mathcal C(\bar I,L^2(\Omega))}\|\partial_x p\|_{L^2(I\times \Omega)}.
%\end{multline*}
Therefore $(\bar y, \bar u)$ is a state-control pair and satisfies the weak formulation \eqref{weakformkdv L4}. The tracking functional (in which we include the $L^4$ part) is weakly lower semi-continuous in $L^4(I\times \Omega)\times L^2(\Omega)$, the control term is weakly-$*$ lower semi-continuous in $\M$. Therefore the state-control pair $(\bar y, \bar u)$ is a minimizer of $J$.
\qquad\end{proof}
%%% Local Variables:
%%% mode: latex
%%% TeX-master: t
%%% End:
